%% Generated by Sphinx.
\def\sphinxdocclass{report}
\documentclass[a4paper,10pt,english,openany,oneside]{sphinxmanual}
\ifdefined\pdfpxdimen
   \let\sphinxpxdimen\pdfpxdimen\else\newdimen\sphinxpxdimen
\fi \sphinxpxdimen=.75bp\relax
\ifdefined\pdfimageresolution
    \pdfimageresolution= \numexpr \dimexpr1in\relax/\sphinxpxdimen\relax
\fi
%% let collapsible pdf bookmarks panel have high depth per default
\PassOptionsToPackage{bookmarksdepth=5}{hyperref}

\PassOptionsToPackage{warn}{textcomp}
\usepackage[utf8]{inputenc}
\ifdefined\DeclareUnicodeCharacter
% support both utf8 and utf8x syntaxes
  \ifdefined\DeclareUnicodeCharacterAsOptional
    \def\sphinxDUC#1{\DeclareUnicodeCharacter{"#1}}
  \else
    \let\sphinxDUC\DeclareUnicodeCharacter
  \fi
  \sphinxDUC{00A0}{\nobreakspace}
  \sphinxDUC{2500}{\sphinxunichar{2500}}
  \sphinxDUC{2502}{\sphinxunichar{2502}}
  \sphinxDUC{2514}{\sphinxunichar{2514}}
  \sphinxDUC{251C}{\sphinxunichar{251C}}
  \sphinxDUC{2572}{\textbackslash}
\fi
\usepackage{cmap}
\usepackage[T1]{fontenc}
\usepackage{amsmath,amssymb,amstext}
\usepackage{babel}



\usepackage{tgtermes}
\usepackage{tgheros}
\renewcommand{\ttdefault}{txtt}



\usepackage[Bjarne]{fncychap}
\usepackage{sphinx}

\fvset{fontsize=auto}
\usepackage{geometry}


% Include hyperref last.
\usepackage{hyperref}
% Fix anchor placement for figures with captions.
\usepackage{hypcap}% it must be loaded after hyperref.
% Set up styles of URL: it should be placed after hyperref.
\urlstyle{same}

\addto\captionsenglish{\renewcommand{\contentsname}{Contents:}}

\usepackage{sphinxmessages}
\setcounter{tocdepth}{4}
\setcounter{secnumdepth}{4}


\title{vipertools}
\date{May 19, 2022}
\release{0.0.1}
\author{Sophia Maedler}
\newcommand{\sphinxlogo}{\vbox{}}
\renewcommand{\releasename}{Release}
\makeindex
\begin{document}

\pagestyle{empty}
\sphinxmaketitle
\pagestyle{plain}
\sphinxtableofcontents
\pagestyle{normal}
\phantomsection\label{\detokenize{index::doc}}

\index{module@\spxentry{module}!vipertools.parse@\spxentry{vipertools.parse}}\index{vipertools.parse@\spxentry{vipertools.parse}!module@\spxentry{module}}

\chapter{parse}
\label{\detokenize{index:parse}}
\sphinxAtStartPar
Contains functions to parse imaging data into a usable formats for downstream pipelines.
\index{parse\_phenix() (in module vipertools.parse)@\spxentry{parse\_phenix()}\spxextra{in module vipertools.parse}}

\begin{fulllineitems}
\phantomsection\label{\detokenize{index:vipertools.parse.parse_phenix}}\pysiglinewithargsret{\sphinxcode{\sphinxupquote{vipertools.parse.}}\sphinxbfcode{\sphinxupquote{parse\_phenix}}}{\emph{\DUrole{n}{phenix\_dir}}, \emph{\DUrole{n}{flatfield\_exported}\DUrole{o}{=}\DUrole{default_value}{True}}, \emph{\DUrole{n}{parallel}\DUrole{o}{=}\DUrole{default_value}{False}}}{}
\sphinxAtStartPar
Function to automatically rename TIFS exported from Harmony into a format where row and well ID as well as Tile position are indicated in the file name.
Example of an exported file name: “Row\{\#\}\_Well\{\#\}\_\{channel\}\_zstack\{\#\}\_r\{\#\}\_c\{\#\}.tif”
\begin{quote}\begin{description}
\item[{Parameters}] \leavevmode\begin{itemize}
\item {} 
\sphinxAtStartPar
\sphinxstyleliteralstrong{\sphinxupquote{phenix\_dir}} \textendash{} Path indicating the exported harmony files to parse.

\item {} 
\sphinxAtStartPar
\sphinxstyleliteralstrong{\sphinxupquote{flatfield\_exported}} \textendash{} boolean indicating if the data was exported from harmony with or without flatfield correction.

\item {} 
\sphinxAtStartPar
\sphinxstyleliteralstrong{\sphinxupquote{parallel}} \textendash{} boolean value indicating if the data parsing should be performed with parallelization or without (CURRENTLY NOT FUNCTIONAL ONLY USE AS FALSE)

\end{itemize}

\end{description}\end{quote}

\end{fulllineitems}

\phantomsection\label{\detokenize{index:module-vipertools.stitch}}\index{module@\spxentry{module}!vipertools.stitch@\spxentry{vipertools.stitch}}\index{vipertools.stitch@\spxentry{vipertools.stitch}!module@\spxentry{module}}

\chapter{stitch}
\label{\detokenize{index:stitch}}
\sphinxAtStartPar
Collection of functions to perform stitching of parsed image Tiffs.
\index{generate\_stitched() (in module vipertools.stitch)@\spxentry{generate\_stitched()}\spxextra{in module vipertools.stitch}}

\begin{fulllineitems}
\phantomsection\label{\detokenize{index:vipertools.stitch.generate_stitched}}\pysiglinewithargsret{\sphinxcode{\sphinxupquote{vipertools.stitch.}}\sphinxbfcode{\sphinxupquote{generate\_stitched}}}{\emph{\DUrole{n}{input\_dir}}, \emph{\DUrole{n}{slidename}}, \emph{\DUrole{n}{pattern}}, \emph{\DUrole{n}{outdir}}, \emph{\DUrole{n}{overlap}}, \emph{\DUrole{n}{stitching\_channel}\DUrole{o}{=}\DUrole{default_value}{\textquotesingle{}Alexa488\textquotesingle{}}}, \emph{\DUrole{n}{crop}\DUrole{o}{=}\DUrole{default_value}{\{\textquotesingle{}bottom\textquotesingle{}: 0, \textquotesingle{}left\textquotesingle{}: 0, \textquotesingle{}right\textquotesingle{}: 0, \textquotesingle{}top\textquotesingle{}: 0\}}}, \emph{\DUrole{n}{plot\_QC}\DUrole{o}{=}\DUrole{default_value}{False}}}{}
\sphinxAtStartPar
Function to generate a scaled down thumbnail of stitched image. Can be used for example to
get a low resolution overview of the scanned region to select areas for exporting high resolution
stitched images.
\begin{quote}\begin{description}
\item[{Parameters}] \leavevmode\begin{itemize}
\item {} 
\sphinxAtStartPar
\sphinxstyleliteralstrong{\sphinxupquote{input\_dir}} \textendash{} Path to the folder containing exported TIF files named with the following nameing convention: “Row\{\#\}\_Well\{\#\}\_\{channel\}\_zstack\{\#\}\_r\{\#\}\_c\{\#\}.tif”.
These images can be generated for example by running the vipertools.parse.parse\_phenix() function.

\item {} 
\sphinxAtStartPar
\sphinxstyleliteralstrong{\sphinxupquote{pattern}} \textendash{} Regex string to identify the naming pattern of the TIFs that should be stitched together.
For example: “Row1\_Well2\_\{channel\}\_zstack3\_r\{row:03\}\_c\{col:03\}.tif”.
All values in \{\} indicate thos which are matched by regex to find all matching tifs.

\item {} 
\sphinxAtStartPar
\sphinxstyleliteralstrong{\sphinxupquote{outdir}} \textendash{} path indicating where the stitched images should be written out

\item {} 
\sphinxAtStartPar
\sphinxstyleliteralstrong{\sphinxupquote{overlap}} \textendash{} value between 0 and 1 indicating the degree of overlap that was used while recording data at the microscope.

\item {} 
\sphinxAtStartPar
\sphinxstyleliteralstrong{\sphinxupquote{name}} \textendash{} string indicating the slidename that is added to the stitched images generated

\item {} 
\sphinxAtStartPar
\sphinxstyleliteralstrong{\sphinxupquote{stitching\_channel}} \textendash{} string indicating the channel name on which the stitching should be calculated. the positions for each tile calculated in this channel will be
passed to the other channels.

\item {} 
\sphinxAtStartPar
\sphinxstyleliteralstrong{\sphinxupquote{crop}} \textendash{} dictionary of the form \{‘top’:0, ‘bottom’:0, ‘left’:0, ‘right’:0\} indicating how many pixels (based on a generated thumbnail,
see vipertools.stitch.generate\_thumbnail) should be cropped from the final image in each indicated dimension. Leave this set to default
if no cropping should be performed.

\end{itemize}

\end{description}\end{quote}

\end{fulllineitems}

\index{generate\_thumbnail() (in module vipertools.stitch)@\spxentry{generate\_thumbnail()}\spxextra{in module vipertools.stitch}}

\begin{fulllineitems}
\phantomsection\label{\detokenize{index:vipertools.stitch.generate_thumbnail}}\pysiglinewithargsret{\sphinxcode{\sphinxupquote{vipertools.stitch.}}\sphinxbfcode{\sphinxupquote{generate\_thumbnail}}}{\emph{\DUrole{n}{input\_dir}}, \emph{\DUrole{n}{pattern}}, \emph{\DUrole{n}{outdir}}, \emph{\DUrole{n}{overlap}}, \emph{\DUrole{n}{name}}, \emph{\DUrole{n}{export\_examples}\DUrole{o}{=}\DUrole{default_value}{False}}}{}
\sphinxAtStartPar
Function to generate a scaled down thumbnail of stitched image. Can be used for example to
get a low resolution overview of the scanned region to select areas for exporting high resolution
stitched images.
\begin{quote}\begin{description}
\item[{Parameters}] \leavevmode\begin{itemize}
\item {} 
\sphinxAtStartPar
\sphinxstyleliteralstrong{\sphinxupquote{input\_dir}} \textendash{} Path to the folder containing exported TIF files named with the following nameing convention: “Row\{\#\}\_Well\{\#\}\_\{channel\}\_zstack\{\#\}\_r\{\#\}\_c\{\#\}.tif”.
These images can be generated for example by running the vipertools.parse.parse\_phenix() function.

\item {} 
\sphinxAtStartPar
\sphinxstyleliteralstrong{\sphinxupquote{pattern}} \textendash{} Regex string to identify the naming pattern of the TIFs that should be stitched together.
For example: “Row1\_Well2\_\{channel\}\_zstack3\_r\{row:03\}\_c\{col:03\}.tif”.
All values in \{\} indicate thos which are matched by regex to find all matching tifs.

\item {} 
\sphinxAtStartPar
\sphinxstyleliteralstrong{\sphinxupquote{outdir}} \textendash{} path indicating where the stitched images should be written out

\item {} 
\sphinxAtStartPar
\sphinxstyleliteralstrong{\sphinxupquote{overlap}} \textendash{} value between 0 and 1 indicating the degree of overlap that was used while recording data at the microscope.

\item {} 
\sphinxAtStartPar
\sphinxstyleliteralstrong{\sphinxupquote{name}} \textendash{} string indicating the slidename that is added to the stitched images generated

\item {} 
\sphinxAtStartPar
\sphinxstyleliteralstrong{\sphinxupquote{export\_examples}} \textendash{} boolean value indicating if individual example tiles should be exported in addition to performing thumbnail generation.

\end{itemize}

\end{description}\end{quote}

\end{fulllineitems}



\renewcommand{\indexname}{Python Module Index}
\begin{sphinxtheindex}
\let\bigletter\sphinxstyleindexlettergroup
\bigletter{v}
\item\relax\sphinxstyleindexentry{vipertools.parse}\sphinxstyleindexpageref{index:\detokenize{module-vipertools.parse}}
\item\relax\sphinxstyleindexentry{vipertools.stitch}\sphinxstyleindexpageref{index:\detokenize{module-vipertools.stitch}}
\end{sphinxtheindex}

\renewcommand{\indexname}{Index}
\printindex
\end{document}