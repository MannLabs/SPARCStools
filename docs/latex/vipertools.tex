%% Generated by Sphinx.
\def\sphinxdocclass{report}
\documentclass[a4paper,10pt,english,openany,oneside]{sphinxmanual}
\ifdefined\pdfpxdimen
   \let\sphinxpxdimen\pdfpxdimen\else\newdimen\sphinxpxdimen
\fi \sphinxpxdimen=.75bp\relax
\ifdefined\pdfimageresolution
    \pdfimageresolution= \numexpr \dimexpr1in\relax/\sphinxpxdimen\relax
\fi
%% let collapsible pdf bookmarks panel have high depth per default
\PassOptionsToPackage{bookmarksdepth=5}{hyperref}

\PassOptionsToPackage{warn}{textcomp}
\usepackage[utf8]{inputenc}
\ifdefined\DeclareUnicodeCharacter
% support both utf8 and utf8x syntaxes
  \ifdefined\DeclareUnicodeCharacterAsOptional
    \def\sphinxDUC#1{\DeclareUnicodeCharacter{"#1}}
  \else
    \let\sphinxDUC\DeclareUnicodeCharacter
  \fi
  \sphinxDUC{00A0}{\nobreakspace}
  \sphinxDUC{2500}{\sphinxunichar{2500}}
  \sphinxDUC{2502}{\sphinxunichar{2502}}
  \sphinxDUC{2514}{\sphinxunichar{2514}}
  \sphinxDUC{251C}{\sphinxunichar{251C}}
  \sphinxDUC{2572}{\textbackslash}
\fi
\usepackage{cmap}
\usepackage[T1]{fontenc}
\usepackage{amsmath,amssymb,amstext}
\usepackage{babel}



\usepackage{tgtermes}
\usepackage{tgheros}
\renewcommand{\ttdefault}{txtt}



\usepackage[Bjarne]{fncychap}
\usepackage{sphinx}

\fvset{fontsize=auto}
\usepackage{geometry}


% Include hyperref last.
\usepackage{hyperref}
% Fix anchor placement for figures with captions.
\usepackage{hypcap}% it must be loaded after hyperref.
% Set up styles of URL: it should be placed after hyperref.
\urlstyle{same}

\addto\captionsenglish{\renewcommand{\contentsname}{Modules:}}

\usepackage{sphinxmessages}
\setcounter{tocdepth}{1}



\title{vipertools}
\date{Jun 10, 2022}
\release{0.0.1}
\author{Sophia Maedler}
\newcommand{\sphinxlogo}{\vbox{}}
\renewcommand{\releasename}{Release}
\makeindex
\begin{document}

\pagestyle{empty}
\sphinxmaketitle
\pagestyle{plain}
\sphinxtableofcontents
\pagestyle{normal}
\phantomsection\label{\detokenize{index::doc}}


\sphinxAtStartPar
Welcome the documentation of VIPERTOOLS!


\chapter{Modules}
\label{\detokenize{pages/modules:module-vipertools.parse}}\label{\detokenize{pages/modules:modules}}\label{\detokenize{pages/modules::doc}}\index{module@\spxentry{module}!vipertools.parse@\spxentry{vipertools.parse}}\index{vipertools.parse@\spxentry{vipertools.parse}!module@\spxentry{module}}

\section{parse}
\label{\detokenize{pages/modules:parse}}
\sphinxAtStartPar
Contains functions to parse imaging data into a usable formats for downstream pipelines.
\index{parse\_phenix() (in module vipertools.parse)@\spxentry{parse\_phenix()}\spxextra{in module vipertools.parse}}

\begin{fulllineitems}
\phantomsection\label{\detokenize{pages/modules:vipertools.parse.parse_phenix}}\pysiglinewithargsret{\sphinxcode{\sphinxupquote{vipertools.parse.}}\sphinxbfcode{\sphinxupquote{parse\_phenix}}}{\emph{\DUrole{n}{phenix\_dir}}, \emph{\DUrole{n}{flatfield\_exported}\DUrole{o}{=}\DUrole{default_value}{True}}, \emph{\DUrole{n}{parallel}\DUrole{o}{=}\DUrole{default_value}{False}}, \emph{\DUrole{n}{WGAbackground}\DUrole{o}{=}\DUrole{default_value}{False}}}{}
\sphinxAtStartPar
Function to automatically rename TIFS exported from Harmony into a format where row and well ID as well as Tile position are indicated in the file name.
Example of an exported file name: “Row\{\#\}\_Well\{\#\}\_\{channel\}\_zstack\{\#\}\_r\{\#\}\_c\{\#\}.tif”
\begin{quote}\begin{description}
\item[{Parameters}] \leavevmode\begin{itemize}
\item {} 
\sphinxAtStartPar
\sphinxstyleliteralstrong{\sphinxupquote{phenix\_dir}} \textendash{} Path indicating the exported harmony files to parse.

\item {} 
\sphinxAtStartPar
\sphinxstyleliteralstrong{\sphinxupquote{flatfield\_exported}} \textendash{} boolean indicating if the data was exported from harmony with or without flatfield correction.

\item {} 
\sphinxAtStartPar
\sphinxstyleliteralstrong{\sphinxupquote{parallel}} \textendash{} boolean value indicating if the data parsing should be performed with parallelization or without (CURRENTLY NOT FUNCTIONAL ONLY USE AS FALSE)

\item {} 
\sphinxAtStartPar
\sphinxstyleliteralstrong{\sphinxupquote{WGAbackground}} \textendash{} export second copy of WGA stains for background correction to improve segmentation. If set to False not performed. Else enter value of the channel
that should be copied and contains the WGA stain.

\end{itemize}

\end{description}\end{quote}

\end{fulllineitems}

\phantomsection\label{\detokenize{pages/modules:module-vipertools.stitch}}\index{module@\spxentry{module}!vipertools.stitch@\spxentry{vipertools.stitch}}\index{vipertools.stitch@\spxentry{vipertools.stitch}!module@\spxentry{module}}

\section{stitch}
\label{\detokenize{pages/modules:stitch}}
\sphinxAtStartPar
Collection of functions to perform stitching of parsed image Tiffs.
\index{generate\_stitched() (in module vipertools.stitch)@\spxentry{generate\_stitched()}\spxextra{in module vipertools.stitch}}

\begin{fulllineitems}
\phantomsection\label{\detokenize{pages/modules:vipertools.stitch.generate_stitched}}\pysiglinewithargsret{\sphinxcode{\sphinxupquote{vipertools.stitch.}}\sphinxbfcode{\sphinxupquote{generate\_stitched}}}{\emph{\DUrole{n}{input\_dir}}, \emph{\DUrole{n}{slidename}}, \emph{\DUrole{n}{pattern}}, \emph{\DUrole{n}{outdir}}, \emph{\DUrole{n}{overlap}}, \emph{\DUrole{n}{stitching\_channel}\DUrole{o}{=}\DUrole{default_value}{\textquotesingle{}Alexa488\textquotesingle{}}}, \emph{\DUrole{n}{crop}\DUrole{o}{=}\DUrole{default_value}{\{\textquotesingle{}bottom\textquotesingle{}: 0, \textquotesingle{}left\textquotesingle{}: 0, \textquotesingle{}right\textquotesingle{}: 0, \textquotesingle{}top\textquotesingle{}: 0\}}}, \emph{\DUrole{n}{plot\_QC}\DUrole{o}{=}\DUrole{default_value}{False}}, \emph{\DUrole{n}{filetype}\DUrole{o}{=}\DUrole{default_value}{{[}\textquotesingle{}.tif\textquotesingle{}{]}}}, \emph{\DUrole{n}{WGAchannel}\DUrole{o}{=}\DUrole{default_value}{None}}, \emph{\DUrole{n}{do\_intensity\_rescale}\DUrole{o}{=}\DUrole{default_value}{True}}, \emph{\DUrole{n}{export\_XML}\DUrole{o}{=}\DUrole{default_value}{True}}}{}
\sphinxAtStartPar
Function to generate a scaled down thumbnail of stitched image. Can be used for example to
get a low resolution overview of the scanned region to select areas for exporting high resolution
stitched images.
\begin{quote}\begin{description}
\item[{Parameters}] \leavevmode\begin{itemize}
\item {} 
\sphinxAtStartPar
\sphinxstyleliteralstrong{\sphinxupquote{input\_dir}} \textendash{} Path to the folder containing exported TIF files named with the following nameing convention: “Row\{\#\}\_Well\{\#\}\_\{channel\}\_zstack\{\#\}\_r\{\#\}\_c\{\#\}.tif”.
These images can be generated for example by running the vipertools.parse.parse\_phenix() function.

\item {} 
\sphinxAtStartPar
\sphinxstyleliteralstrong{\sphinxupquote{pattern}} \textendash{} Regex string to identify the naming pattern of the TIFs that should be stitched together.
For example: “Row1\_Well2\_\{channel\}\_zstack3\_r\{row:03\}\_c\{col:03\}.tif”.
All values in \{\} indicate thos which are matched by regex to find all matching tifs.

\item {} 
\sphinxAtStartPar
\sphinxstyleliteralstrong{\sphinxupquote{outdir}} \textendash{} path indicating where the stitched images should be written out

\item {} 
\sphinxAtStartPar
\sphinxstyleliteralstrong{\sphinxupquote{overlap}} \textendash{} value between 0 and 1 indicating the degree of overlap that was used while recording data at the microscope.

\item {} 
\sphinxAtStartPar
\sphinxstyleliteralstrong{\sphinxupquote{name}} \textendash{} string indicating the slidename that is added to the stitched images generated

\item {} 
\sphinxAtStartPar
\sphinxstyleliteralstrong{\sphinxupquote{stitching\_channel}} \textendash{} string indicating the channel name on which the stitching should be calculated. the positions for each tile calculated in this channel will be
passed to the other channels.

\item {} 
\sphinxAtStartPar
\sphinxstyleliteralstrong{\sphinxupquote{crop}} \textendash{} dictionary of the form \{‘top’:0, ‘bottom’:0, ‘left’:0, ‘right’:0\} indicating how many pixels (based on a generated thumbnail,
see vipertools.stitch.generate\_thumbnail) should be cropped from the final image in each indicated dimension. Leave this set to default
if no cropping should be performed.

\item {} 
\sphinxAtStartPar
\sphinxstyleliteralstrong{\sphinxupquote{plot\_QC}} \textendash{} boolean value indicating if QC plots should be generated

\item {} 
\sphinxAtStartPar
\sphinxstyleliteralstrong{\sphinxupquote{filetype}} \textendash{} list containing any of {[}“.tif”, “.ome.zarr”, “.ome.tif”{]} defining to which type of file the stiched results should be written. If more than one
element all export types will be generated in the same output directory.

\item {} 
\sphinxAtStartPar
\sphinxstyleliteralstrong{\sphinxupquote{WGAchannel}} \textendash{} string indicating the name of the WGA channel in case an illumination correction should be performed on this cahhenl

\item {} 
\sphinxAtStartPar
\sphinxstyleliteralstrong{\sphinxupquote{do\_intensity\_rescale}} \textendash{} boolean value indicating if the rescale\_p1\_P99 function should be applied before stitching or not.

\item {} 
\sphinxAtStartPar
\sphinxstyleliteralstrong{\sphinxupquote{export\_XML}} \textendash{} boolean value. If true than an xml is exported when writing to .tif which allows for the import into BIAS.

\end{itemize}

\end{description}\end{quote}

\end{fulllineitems}

\index{generate\_thumbnail() (in module vipertools.stitch)@\spxentry{generate\_thumbnail()}\spxextra{in module vipertools.stitch}}

\begin{fulllineitems}
\phantomsection\label{\detokenize{pages/modules:vipertools.stitch.generate_thumbnail}}\pysiglinewithargsret{\sphinxcode{\sphinxupquote{vipertools.stitch.}}\sphinxbfcode{\sphinxupquote{generate\_thumbnail}}}{\emph{\DUrole{n}{input\_dir}}, \emph{\DUrole{n}{pattern}}, \emph{\DUrole{n}{outdir}}, \emph{\DUrole{n}{overlap}}, \emph{\DUrole{n}{name}}, \emph{\DUrole{n}{stitching\_channel}\DUrole{o}{=}\DUrole{default_value}{\textquotesingle{}DAPI\textquotesingle{}}}, \emph{\DUrole{n}{export\_examples}\DUrole{o}{=}\DUrole{default_value}{False}}, \emph{\DUrole{n}{do\_intensity\_rescale}\DUrole{o}{=}\DUrole{default_value}{True}}}{}
\sphinxAtStartPar
Function to generate a scaled down thumbnail of stitched image. Can be used for example to
get a low resolution overview of the scanned region to select areas for exporting high resolution
stitched images.
\begin{quote}\begin{description}
\item[{Parameters}] \leavevmode\begin{itemize}
\item {} 
\sphinxAtStartPar
\sphinxstyleliteralstrong{\sphinxupquote{input\_dir}} \textendash{} Path to the folder containing exported TIF files named with the following nameing convention: “Row\{\#\}\_Well\{\#\}\_\{channel\}\_zstack\{\#\}\_r\{\#\}\_c\{\#\}.tif”.
These images can be generated for example by running the vipertools.parse.parse\_phenix() function.

\item {} 
\sphinxAtStartPar
\sphinxstyleliteralstrong{\sphinxupquote{pattern}} \textendash{} Regex string to identify the naming pattern of the TIFs that should be stitched together.
For example: “Row1\_Well2\_\{channel\}\_zstack3\_r\{row:03\}\_c\{col:03\}.tif”.
All values in \{\} indicate thos which are matched by regex to find all matching tifs.

\item {} 
\sphinxAtStartPar
\sphinxstyleliteralstrong{\sphinxupquote{outdir}} \textendash{} path indicating where the stitched images should be written out

\item {} 
\sphinxAtStartPar
\sphinxstyleliteralstrong{\sphinxupquote{overlap}} \textendash{} value between 0 and 1 indicating the degree of overlap that was used while recording data at the microscope.

\item {} 
\sphinxAtStartPar
\sphinxstyleliteralstrong{\sphinxupquote{name}} \textendash{} string indicating the slidename that is added to the stitched images generated

\item {} 
\sphinxAtStartPar
\sphinxstyleliteralstrong{\sphinxupquote{export\_examples}} \textendash{} boolean value indicating if individual example tiles should be exported in addition to performing thumbnail generation.

\item {} 
\sphinxAtStartPar
\sphinxstyleliteralstrong{\sphinxupquote{do\_intensity\_rescale}} \textendash{} boolean value indicating if the rescale\_p1\_P99 function should be applied before stitching or not.

\end{itemize}

\end{description}\end{quote}

\end{fulllineitems}



\chapter{Tutorials}
\label{\detokenize{pages/tutorials:tutorials}}\label{\detokenize{pages/tutorials::doc}}

\section{Parsing and Stitching Data from Opera Phenix}
\label{\detokenize{pages/tutorials:parsing-and-stitching-data-from-opera-phenix}}
\sphinxAtStartPar
First you need to export your data from harmony and rename the path to eliminate any spaces in the name.
Then you can run the following script to parses and stitch your data.
\sphinxSetupCaptionForVerbatim{example script for parsing and stitching phenix data}
\def\sphinxLiteralBlockLabel{\label{\detokenize{pages/tutorials:id1}}}
\begin{sphinxVerbatim}[commandchars=\\\{\}]
\PYG{c+c1}{\PYGZsh{}import relevant libraries}
\PYG{k+kn}{import} \PYG{n+nn}{os}
\PYG{k+kn}{from} \PYG{n+nn}{vipertools}\PYG{n+nn}{.}\PYG{n+nn}{parse} \PYG{k+kn}{import} \PYG{n}{parse\PYGZus{}phenix}
\PYG{k+kn}{from} \PYG{n+nn}{vipertools}\PYG{n+nn}{.}\PYG{n+nn}{stitch} \PYG{k+kn}{import} \PYG{n}{generate\PYGZus{}stitched}

\PYG{c+c1}{\PYGZsh{}parse image data}
\PYG{n}{path} \PYG{o}{=} \PYG{l+s+s2}{\PYGZdq{}}\PYG{l+s+s2}{path to exported harmony project without any spaces}\PYG{l+s+s2}{\PYGZdq{}}
\PYG{n}{parse\PYGZus{}phenix}\PYG{p}{(}\PYG{n}{path}\PYG{p}{,} \PYG{n}{flatfield\PYGZus{}exported} \PYG{o}{=} \PYG{k+kc}{True}\PYG{p}{,} \PYG{n}{parallel} \PYG{o}{=} \PYG{k+kc}{False}\PYG{p}{)}

\PYG{c+c1}{\PYGZsh{}define important information for your slide that you want to stitch}

\PYG{c+c1}{\PYGZsh{} the code below needs to be run for each slide contained in the imaging experiment!}
\PYG{c+c1}{\PYGZsh{} Can be put into a loop for example to automate this or also can be subset to seperate}
\PYG{c+c1}{\PYGZsh{} jobs to run on for example the hpc}

\PYG{n}{input\PYGZus{}dir} \PYG{o}{=} \PYG{n}{os}\PYG{o}{.}\PYG{n}{path}\PYG{o}{.}\PYG{n}{join}\PYG{p}{(}\PYG{n}{path}\PYG{p}{,} \PYG{l+s+s2}{\PYGZdq{}}\PYG{l+s+s2}{parsed\PYGZus{}images}\PYG{l+s+s2}{\PYGZdq{}}\PYG{p}{)}
\PYG{n}{slidename} \PYG{o}{=} \PYG{l+s+s2}{\PYGZdq{}}\PYG{l+s+s2}{Slide1}\PYG{l+s+s2}{\PYGZdq{}}
\PYG{n}{outdir} \PYG{o}{=} \PYG{n}{os}\PYG{o}{.}\PYG{n}{path}\PYG{o}{.}\PYG{n}{join}\PYG{p}{(}\PYG{n}{path}\PYG{p}{,} \PYG{l+s+s2}{\PYGZdq{}}\PYG{l+s+s2}{stitched}\PYG{l+s+s2}{\PYGZdq{}}\PYG{p}{,} \PYG{n}{slidename}\PYG{p}{)}
\PYG{n}{overlap} \PYG{o}{=} \PYG{l+m+mf}{0.1} \PYG{c+c1}{\PYGZsh{}adjust in case your data was aquired with another overlap}

\PYG{c+c1}{\PYGZsh{}define parameters to find correct slide in experiment folder}
\PYG{n}{row} \PYG{o}{=} \PYG{l+m+mi}{1}
\PYG{n}{well} \PYG{o}{=} \PYG{l+m+mi}{1}
\PYG{n}{zstack\PYGZus{}value} \PYG{o}{=} \PYG{l+m+mi}{1}

\PYG{c+c1}{\PYGZsh{}define on which channel should be stitched}
\PYG{n}{stitching\PYGZus{}channel} \PYG{o}{=} \PYG{l+s+s2}{\PYGZdq{}}\PYG{l+s+s2}{Alexa647}\PYG{l+s+s2}{\PYGZdq{}}
\PYG{n}{output\PYGZus{}filetype} \PYG{o}{=} \PYG{p}{[}\PYG{l+s+s2}{\PYGZdq{}}\PYG{l+s+s2}{.tif}\PYG{l+s+s2}{\PYGZdq{}}\PYG{p}{,} \PYG{l+s+s2}{\PYGZdq{}}\PYG{l+s+s2}{ome.zarr}\PYG{l+s+s2}{\PYGZdq{}}\PYG{p}{]} \PYG{c+c1}{\PYGZsh{}one of .tif, .ome.tif, .ome.zarr (can pass several if you want to generate all filetypes)}

\PYG{c+c1}{\PYGZsh{}adjust cropping parameter}
\PYG{n}{crop} \PYG{o}{=} \PYG{p}{\PYGZob{}}\PYG{l+s+s1}{\PYGZsq{}}\PYG{l+s+s1}{top}\PYG{l+s+s1}{\PYGZsq{}}\PYG{p}{:}\PYG{l+m+mi}{0}\PYG{p}{,} \PYG{l+s+s1}{\PYGZsq{}}\PYG{l+s+s1}{bottom}\PYG{l+s+s1}{\PYGZsq{}}\PYG{p}{:}\PYG{l+m+mi}{0}\PYG{p}{,} \PYG{l+s+s1}{\PYGZsq{}}\PYG{l+s+s1}{left}\PYG{l+s+s1}{\PYGZsq{}}\PYG{p}{:}\PYG{l+m+mi}{0}\PYG{p}{,} \PYG{l+s+s1}{\PYGZsq{}}\PYG{l+s+s1}{right}\PYG{l+s+s1}{\PYGZsq{}}\PYG{p}{:}\PYG{l+m+mi}{0}\PYG{p}{\PYGZcb{}}  \PYG{c+c1}{\PYGZsh{}this does no cropping}
\PYG{c+c1}{\PYGZsh{}crop = \PYGZob{}\PYGZsq{}top\PYGZsq{}:72, \PYGZsq{}bottom\PYGZsq{}:52, \PYGZsq{}left\PYGZsq{}:48, \PYGZsq{}right\PYGZsq{}:58\PYGZcb{} \PYGZsh{}this is good default values for an entire PPS slide with cell culture samples imaged with my protocol}

\PYG{c+c1}{\PYGZsh{}create output directory if it does not exist}
\PYG{k}{if} \PYG{o+ow}{not} \PYG{n}{os}\PYG{o}{.}\PYG{n}{path}\PYG{o}{.}\PYG{n}{exists}\PYG{p}{(}\PYG{n}{outdir}\PYG{p}{)}\PYG{p}{:}
    \PYG{n}{os}\PYG{o}{.}\PYG{n}{makedirs}\PYG{p}{(}\PYG{n}{outdir}\PYG{p}{)}

\PYG{c+c1}{\PYGZsh{}define patter to recognize which slide should be stitched}
\PYG{c+c1}{\PYGZsh{}remember to adjust the zstack value if you aquired zstacks and want to stitch a speciifc one in the parameters above}

\PYG{n}{pattern} \PYG{o}{=} \PYG{l+s+s2}{\PYGZdq{}}\PYG{l+s+s2}{Row}\PYG{l+s+s2}{\PYGZdq{}}\PYG{o}{+} \PYG{n+nb}{str}\PYG{p}{(}\PYG{n}{row}\PYG{p}{)} \PYG{o}{+} \PYG{l+s+s2}{\PYGZdq{}}\PYG{l+s+s2}{\PYGZus{}}\PYG{l+s+s2}{\PYGZdq{}} \PYG{o}{+} \PYG{l+s+s2}{\PYGZdq{}}\PYG{l+s+s2}{Well}\PYG{l+s+s2}{\PYGZdq{}} \PYG{o}{+} \PYG{n+nb}{str}\PYG{p}{(}\PYG{n}{well}\PYG{p}{)} \PYG{o}{+} \PYG{l+s+s2}{\PYGZdq{}}\PYG{l+s+s2}{\PYGZus{}}\PYG{l+s+si}{\PYGZob{}channel\PYGZcb{}}\PYG{l+s+s2}{\PYGZus{}}\PYG{l+s+s2}{\PYGZdq{}}\PYG{o}{+}\PYG{l+s+s2}{\PYGZdq{}}\PYG{l+s+s2}{zstack}\PYG{l+s+s2}{\PYGZdq{}}\PYG{o}{+}\PYG{n+nb}{str}\PYG{p}{(}\PYG{n}{zstack\PYGZus{}value}\PYG{p}{)}\PYG{o}{+}\PYG{l+s+s2}{\PYGZdq{}}\PYG{l+s+s2}{\PYGZus{}r}\PYG{l+s+si}{\PYGZob{}row:03\PYGZcb{}}\PYG{l+s+s2}{\PYGZus{}c}\PYG{l+s+si}{\PYGZob{}col:03\PYGZcb{}}\PYG{l+s+s2}{.tif}\PYG{l+s+s2}{\PYGZdq{}}
\PYG{n}{generate\PYGZus{}stitched}\PYG{p}{(}\PYG{n}{input\PYGZus{}dir}\PYG{p}{,}
                    \PYG{n}{slidename}\PYG{p}{,}
                    \PYG{n}{pattern}\PYG{p}{,}
                    \PYG{n}{outdir}\PYG{p}{,}
                    \PYG{n}{overlap}\PYG{p}{,}
                    \PYG{n}{crop} \PYG{o}{=} \PYG{n}{crop} \PYG{p}{,}
                    \PYG{n}{stitching\PYGZus{}channel} \PYG{o}{=} \PYG{n}{stitching\PYGZus{}channel}\PYG{p}{,}
                    \PYG{n}{filetype} \PYG{o}{=} \PYG{n}{output\PYGZus{}filetype}\PYG{p}{)}
\end{sphinxVerbatim}


\renewcommand{\indexname}{Python Module Index}
\begin{sphinxtheindex}
\let\bigletter\sphinxstyleindexlettergroup
\bigletter{v}
\item\relax\sphinxstyleindexentry{vipertools.parse}\sphinxstyleindexpageref{pages/modules:\detokenize{module-vipertools.parse}}
\item\relax\sphinxstyleindexentry{vipertools.stitch}\sphinxstyleindexpageref{pages/modules:\detokenize{module-vipertools.stitch}}
\end{sphinxtheindex}

\renewcommand{\indexname}{Index}
\printindex
\end{document}